\newcommand\Colorhref[3][color1]{\href{#2}{\color{#1}\underline{#3}}}
\documentclass[10pt,a4paper]{moderncv}

\usepackage{moderntimeline}
\usepackage[french]{babel}
\usepackage[utf8x]{inputenc}
\usepackage[T1]{fontenc}

%\renewcommand{\FrenchLabelItem}{\textcolor{blue}{$\circ$}}

\moderncvstyle{classic}
\moderncvcolor{blue}
\tlmaxdates{2004}{2017}
\tlwidth{0.8ex}
\tltext{\tiny}
\tltextstart[base]{\scriptsize}
% adjust the page margins
\usepackage[scale=0.90]{geometry}
\setlength{\hintscolumnwidth}{2.5cm}           % if you want to change the width of the column with the dates

% personal data
\firstname{Guillaume}
\familyname{Lhermenier}
\title{Expert en rien et curieux de tout qui aime \newline
apprendre et échanger. \newline 
Et si ca peut être dans un environnement cool...}
\address{21 avenue du Bel Air}{77340 Pontault-Combault}
\mobile{06-21-67-54-21}
% \phone{xx-xx-xx-xx-xx}
\email{guillaume.lhermenier@gmail.com}
\social[github][github.com/gocho1]{github.com/gocho1}
\social[linkedin][www.linkedin.com/in/guillaumelhermenier/]{Guillaume Lhermenier}
\extrainfo{33 ans - Permis B}
%\photo[60pt][0.2pt]{photo_cv}
\quote{""}
\quote{"It’s fine to celebrate success but it is more important to heed the lessons of failure." -- Bill Gates}

\begin{document}

\makecvtitle

\section{Expériences professionnelles}

\tlcventry{2017}{0}{Developpeur curieux intéressé par l'automatisation}{BNP Paribas}{Montreuil}{}
{
\begin{itemize}
 \item Réalisation de mécanismes d'ingestion data génériques pour les différentes entités du groupe
  \begin{itemize}
   \item Diversité technique importante sur les différents mécanismes : \textbf{Kafka}, Java, Scala, Angular, Springboot, IBM WAS Liberty Core, Stack Elastic/Logstash/Kibana
   \item Mise en place de l'intégration et déploiement continu des projets : Jenkins, \textbf{Ansible}, CA ARA, \textbf{Docker}
   \item \textbf{Automatisation} de tests (unitaires, intégration, non régression, performances) dans une approche multi-composants
  \end{itemize}
 \item Contexte international et agile : Echanges écrits et oraux en anglais dans une équipe multinationale (Belgique flamande, Italie, France) 
\end{itemize}
}

\tlcventry{2016}{2017}{DevOps débutant}{BNP Paribas}{Montreuil}{1 an 1 mois}
{
\begin{itemize}
 \item Appui technique et "process internes" auprès des équipes de developpeurs de divers projets
  \begin{itemize}
   \item Mise en place d'outils pour faciliter le quotidien des équipes (gestionnaire de mots de passe centralisé, Suivi de production automatisé) 
   \item \textbf{"\'{E}vangelisation"} sur l'automatisation, édition de bonnes pratiques 
   \item \textbf{DevOps} : Rapprochement des développeurs et des équipes de production, assistance aux développeurs afin de rendre les applications déployables automatiquement, proprement et de manière intégrée au SI
   \item Nouvelles approches technologiques : \textbf{Docker}, Kubernetes
  \end{itemize}
\end{itemize}
}


\tlcventry{2016}{2016}{Chef de projet BI}{BNP Paribas}{Montreuil}{10 mois}
{
\begin{itemize}
 \item Mise en place d'une application web de reporting sur les virement internationaux des clients Entreprises et Institutions (CIB) \newline
	\textit{Specifications fonctionnelles et techniques, Design BDD, pilotage, réalisation dictionnaire de données Liberty Pilot}
 \item Découverte de nouvelles technologies : JQuery, D3.js, Liberty Pilot \& Insight 
\end{itemize}
}



\tlcventry{2015}{2016}{Responsable d'applications BI}{BNP Paribas}{Montreuil}{1 an}
{
\begin{itemize}
 \item Maintenance opérationnelle et technique de plusieurs applications de reportings \newline
 	\textit{Suivi des évolutions et de la production, pilotage planning et budgétaire, pilotage de la TMA}
 \item Rationalisation du processus de releases des applications du domaine Reporting : Intégration Continue (SVN, Jenkins, Sonar, Nexus) et Déploiement Continu (Pilote sur CA ARA)  
\end{itemize}
}


\tlcventry{2013}{2014}{Chef de projet BI}{BNP Paribas}{Montreuil}{1 an 2 mois}
{
\begin{itemize}
 \item Mise en place d'une application de reporting destinées aux conseillers d'Affaires Entreprises et Retail de la Banque De Détails France (BDDF) pour le suivi des PME. \newline
   \textit{Note de cadrage de la solution technique, réalisation des spécifications fonctionnelles et techniques (datamart, reportings), pilotage de la réalisation}
 \item Technologies : Cognos BI 10.2, Teradata (12,14), IBM WAS 8.5, Jave J2EE
\end{itemize}
}

\tlcventry{2010}{2013}{Responsable d'applications BI}{BNP Paribas}{Montreuil}{3 ans}
{
\begin{itemize}
 \item Responsable d'applications BI destinées aux conseillers d'Affaires des réseaux Entreprises et Retail de la Banque de Détail France (BDDF) \newline
	\textit{Cadrage des évolutions, suivi des réalisations de la TMA, pilotage budgétaire des applications}
 \item Technologies : Business Objects (Xir2, XI3), Teradata (v2r6, 12), IBM WAS ND (6,7)
\end{itemize}
}

\tlcventry{2009}{2010}{Responsable Pôle de compétences Access}{Allianz Group}{La Défense}{1 an}
{
\begin{itemize}
  \item Prise en charge des analyses et développement d’applications multiutilisateurs : Portefeuille de contrats obsèques (Caton), Suivi des contrôles quotidiens d'un gestionnaire VIE, Suivi du plan de contrôles locaux DDOV
  \item Maintenance applicative des outils existants, formation et accompagnement des utilisateurs : cartographie des processus de gestion manuelle DR Paris, Mise en place d'un process de déploiement/mise à jour des applications
\end{itemize}
}

\tlcventry{2007}{2009}{Assistant chef de projet Maitrise d'Ouvrage (Alternance)}{AGF}{La Défense}{2 ans}
{
\begin{itemize}
 \item Etude d'un projet d'industrialisation de jeux d'essais pour le domaine de la recette fonctionnelle
 \item Refonte du référentiel des recettes applicatives : Etablissement de modèles de fichiers de suivi, Supervision des réalisations et suivi d’avancement, conseil méthodologique
 \item Adhésion à un pôle de compétence Access (études et développement d'applications pour la direction VIE) : Outil de suivi de formation de nouveaux collaborateurs (OTP), Outil de suivi de l'opération de sécurisation du portefeuille (DDAC)
\end{itemize}
}

\tlcventry{2006}{2007}{Analyste Developpeur (Alternance L3 MIAGE)}{Carrefour pour le compte d'Unilog}{Evry}{1 an}
{Activités de recettes applicatives (rédaction dossiers, plans de test, reporting)
Participation à la Tierce Recette Applicative d'un site de e-commerce pour un grand compte de la distribution}

\bigskip

\section{Formation}

\tlcventry{2006}{2009}{Cursus MIAGE (Licence, Master)}{Université Evry Val d'Essonne}{}{3 ans}{}
\tlcventry{2004}{2006}{DUT Informatique de Gestion}{IUT Fontainebleau}{}{2 ans}{}

\section{Compétences}

\cvitem{Développement}{Java, Shell Scripting, notions de Scala, Python}{}{}
\cvitem{Automatisation}{\textbf{Ansible}}{}{}	
\cvitem{Conteneurisation}{\textbf{Docker}, notions de LXC}{}{}
\cvitem{Système}{\textbf{GNU/Linux} (RHEL, CentOS)}{}{}
\cvitem{Divers}{intégration continue}{}{}

\section{Langues}

\cvlanguage{Anglais}{Courant}{Pratiqué quotidiennement}
\cvlanguage{Allemand}{Notions}{\'{E}tudié 7 ans, très peu pratiqué}

\section{Centres d'interêt}

\cvitem{Sport}{badminton, football}
\cvitem{Autres}{Lecture, nouvelles technologies}

\end{document}

